\documentclass[sigconf]{acmart}
%\settopmatter{printacmref=false} % Removes citation information below abstract
\renewcommand\footnotetextcopyrightpermission[1]{} % removes footnote with conference information in first column
\pagestyle{plain} % removes running headers

\usepackage{booktabs} % For formal tables
\usepackage{listings}
\usepackage{hyperref}
\hypersetup{
    colorlinks=true,
    linkcolor=blue,
    filecolor=magenta,      
    urlcolor=cyan,
}

\begin{document}
\title{Data analysis and visualization of Indian Premier League(IPL) matches}

\author{Sandeep Khandelwal}
\email{skhande@iu.edu}

\author{Abhishek Gupta}
\email{abhigupt@iu.edu}

\begin{abstract}

This exploratory analysis will help us visualize the best players in previous
IPL matches both bowlers and batsman. It will help use visualize best 
teams during past matches. There will also be a breakdown on whether
home pitch was favorable in terms of wins. It will also depict how the players 
performed and eventually led to a match winning effort. For example if 
a team has best players but they never performed together in a single match.
If they all play well, it will result in a match win. Hence, these facts can be used to
predict the winners in upcoming match. Other analysis can be done are players by
number of run, wickets, maximum number of six's, four's etc. Man of the match
by each season. Average runs scored by each team over by over. It may also
have analysis on poorly performing players for each team. Poor players 
can be based on their bowling or batting scores. 
   
\end{abstract}

\keywords{ipl, analysis, python, packages, bowling, batting}

\maketitle

\section{introduction}
We would like to analyze IPL match for last decade for all IPL matches held. This analysis
will be done on ball by ball data from previous IPL matches. Some important visualization
we plan to create are
\begin {itemize}
\item top batsman
\item top bowlers
\item best team by year
\item players by max six's
\item players by maximum fours's
\item players by maximum man of matches
\item hight run grossers
\item maximum wicket takers
\item team performance by venue 
\end {itemize}

These are key visualizations we plan to make but may not be limited to these 
visualization and may have more added to the list above. 

\section{technology}
We plan to use python as programming language and use D3 or Plotly as charting
library. We plan to incorporate all charts in a jupyter notebook. We may also
use other python modules like matplotlib and pandas if needed. 


\section{Related Work} \label{relwork}
\begin {itemize}
\item
iplt20 website - \cite{www-iplt20} 
This website display insights related to each IPL match.
TODO

\item
Reflecting Against Perception: Data Analysis of IPL Batsman \cite{kumar2014reflecting}.
TODO

\end {itemize}

\section{Further Enhancement} \label{enhancements}
Future enhancements can be done interms of building predictive model to predict the winning team


%\input{similartechnologies}

%\input{conclusion}

%\end{document}  % This is where a 'short' article might terminate


\section{Acknowledgements}
 The authors thank Prof. YY Ahn for his technical guidance. The
 authors would also like to thank TAs of Data Visualization class for their valued
 support. 

\section{Repo} 
 All project and report document can be found at \href{https://github.com/abhishek8gupta/dviz-project-summer2018/blob/master/report/main.pdf}{github project}.

\nocite{*}
\bibliographystyle{unsrt}
\bibliography{references}

\end{document}
