\documentclass[sigconf]{acmart}
%\settopmatter{printacmref=false} % Removes citation information below abstract
\renewcommand\footnotetextcopyrightpermission[1]{} % removes footnote with conference information in first column
\pagestyle{plain} % removes running headers

\usepackage{booktabs} % For formal tables
\usepackage{listings}
\usepackage{hyperref}
\usepackage{placeins}
\hypersetup{
    colorlinks=true,
    linkcolor=blue,
    filecolor=magenta,      
    urlcolor=cyan,
}

\begin{document}
\title{Data analysis and visualization of Indian Premier League(IPL) matches}

\author{Sandeep Khandelwal}
\email{skhande@iu.edu}

\author{Abhishek Gupta}
\email{abhigupt@iu.edu}

\begin{abstract}

This exploratory analysis will help us visualize the best players and team in previous
IPL matches. There will also be a breakdown on whether
home pitch was favorable in terms of wins. It will also depict how the players 
performed and eventually led to a match winning effort. For example if 
a team has best players but they never performed together in a single match.
If they all play well, it will result in a match win. Hence, these facts can be used to
predict the winners in upcoming match. Other analysis can be done are players by
number of run, wickets, maximum number of six's, four's etc. Man of the match
by each season. Average runs scored by each team over by over. It may also
have analysis on poorly performing players for each team. Poor players 
can be based on their bowling or batting scores. 
   
\end{abstract}

\keywords{ipl, analysis, python, packages, bowling, batting, fielding, match strategy, T20, lbw, wickets, runs, bowled, stumped}

\maketitle

\section{introduction}
We would like to analyze IPL match for last decade for all IPL matches played. This analysis
will be done on ball by ball data available from previous IPL matches. This analysis will help 
visualize the data as well from various dimensions of a match, tournament, location, player etc.

\subsection{Convincing Motivation}
The official T20 \cite{www-iplt20} site displays data in a very rudimentary way. It doesn't
provide good visualization of IPL data. If you visit the page for historical data it shows
cross tab data accross various dimensions. It shows year wise data from 2008 which cuts
across various dimensions like runs, players, wickets etc. 

\begin{figure}[htbp]
\centering
\fbox{\includegraphics[width=\linewidth]{images/iplt20.png}}
\caption{IPL T20 official webpage}
\label{fig:iplt20}
\end{figure}

The data for all the historical matches is very monotonous and tabular. Tabular representation
make it hard to understand and see any type of pattern or anomaly. It also doesn't allow uses to 
filter the data or aggregate the data for a given range of dates or select a subset of teams. For 
example one cannot select top batsman from 3 given teams or top bowler from a subset of teams.
The data may be sometimes helpful when there is a T20 tournament and only few teams are
participating. Another important aspect left out is analysis based on location. As we can see from 
available leaders it doesn't show any report based on location. This may sometimes help you
to answer some questions like how two teams have performed in past on a given location or if 
this venue has favored home team or if this is a good batting location or a bowling location. These
questions really help to predict the outcome of the match. 
Overall, it lacks
\begin{itemize}
\item monotonous
\item filter support
\item aggregation
\item trend / pattern
\item location
\end{itemize}

\subsection{Existing Work}
For IPL data analysis, I explored few of the existing works to see what kind of insights
they provide and what are the drawbacks of this data analysis in terms of visualizing this data.

\subsubsection{IPL T20 Official Webpage}

As described above, IPL T20\cite{www-iplt20} official website provides good insight on the 
data for previous year's IPL tournament but fails to provide good visualization. The visualization
provided by this web page is very monotonous and boring. For example, the following figure 
shows very monotonous tabular data.

\begin{figure}[htbp]
\centering
\fbox{\includegraphics[width=\linewidth]{images/t20highscorers.png}}
\caption{IPL T20 High Run Scorers}
\label{fig:t20highscorers}
\end{figure}

\subsubsection{CricViz Analysis}
CricViz analysis \cite{www-cricviz} provides visualization using line chart and tabular data.
The line charts show match history for all IPL teams and briefly discusses for the teams
based on their IPL history. Further provides analysis of delivery and length of the bowlers.
These charts lack depth of analysis of the data as well as cool visualization of the data.

 \begin{figure}[htbp]
\centering
\fbox{\includegraphics[width=\linewidth]{images/teamAnalysis}}
\caption{IPL history of various teams}
\label{fig:teamAnalysis}
\end{figure}

\begin{figure}[htbp]
\centering
\fbox{\includegraphics[width=\linewidth]{images/deliveryType}}
\caption{Delivery Type Analysis}
\label{fig:deliveryType}
\end{figure}

\begin{figure}[htbp]
\centering
\fbox{\includegraphics[width=\linewidth]{images/boundarySucess}}
\caption{Analysis of IPL T20 matches}
\label{fig:boundarySucess}
\end{figure}

\subsubsection{Data visualization hacknight: using Pandas for analyzing IPL T20 data}
This team tries to visualize \cite{www-fifthelephant} the IPL T20 data using some kind of heat maps. 
Again this lacks most the needed insights.

\begin{figure}[htbp]
\centering
\fbox{\includegraphics[width=\linewidth]{images/heatmaps}}
\caption{Analysis of IPL T20 data using pandas}
\label{fig:heatmaps}
\end{figure}

\subsubsection{Reflecting Against Perception: Data Analysis of IPL Batsman }
This paper \cite{kumar2014reflecting} discusses various formulas to come up with
a predictive model for IPL matches for example batsman impacting score, run impact
score, strike rate impact store etc which is further used to compute the batsman rank etc. 

\begin{figure}[htbp]
\centering
\fbox{\includegraphics[width=\linewidth]{images/battingRNK}}
\caption{Data Analysis of IPL Batsman}
\label{fig:battingRNK}
\end{figure}

This analysis is completely focussed on building predictive model and also predict the batsman
performance based on this model. Completely lacks visualization.

\subsubsection{Analysis of IPL T20 matches with yorkr templates}
This blog \cite{www-rbloggers-ipl} tries analyze IPL data and creates modules for 
various kinds of reports like batsman cumulative strike rate, batsman dismissal etc
But this blog fails to provide any details on visualization techniques. The blog mainly
focuses on various methods to query the data. It also creates data frame to query the data directly.
The methods mainly focus on batsman and bowler statistics functions. 
These include
\begin{itemize}
\item Runs vs deliveries
\item Batsman 4s and 6s
\item Batsman dismissals
\item Runs vs Strike rate
\item Batsman Moving Average
\item Batsman cumulative average
\item Batsman cumulative strike rate
\item Batsman runs against oppositions
\item Batsman runs vs venue
\item Batsman runs predict
\item Bowler mean runs conceded
\item Bowler Moving Average
\item Bowler cumulative average wickets
\item Bowler wicket plot
\item Bowler wicket against opposition
\item Bowler wicket at cricket grounds
\end{itemize}

These methods are close to the kind of reports and visualization
we  should be doing for IPL T20 matches but doesn't provide details
of any suggested visualization methods.

\subsubsection{Contribution}
As a part of this analysis of IPL data, our goal here is to provide
more meaningful visualization to this data for example if I want to 
see how tournament progressed in year 2009 or which venues 
are high score venues. Map kind of visualization will be much better than showing
just a tabular data. Or showing Gantt's chart make more sense for 
showing tournament progress rather than showing tabular data. 
As a part of this analysis, we have tried to look at the statistics shown during a 
live match and created similar and more meaningful visualizations for historical data. 
Another example could be on how to show highest run scorers? In all above
methods, the output is demonstrated using tabular data which is good but doesn't 
visualize the data correctly. Similar is the case of bowlers who took highest wickets. 
For these two use cases I have used treemap which make more sense which represents
a larger tile for the player who scored highest or took maximum wickets along with the 
team to which he belongs and then reducing the tile size for the players who scored less
and so on.  

\section{Data and Methods}
\subsubsection{ideas, sketches, prototypes}
The idea here is to show two type of drill downs of the tournament, one is location based
and second it year based. For example, if we open a map we should be able to select a location
and further visualize the data based on location as a dimension. Also, we should be able to
further apply additional filters like team, year etc. Another visualization dimension could be 
year of the tournament where any visualization could be filtered based on a particular year 
or a subset of years. These a two primary approaches to start visualizing data. Further, we 
should be able to do analysis on match level or player level statistics. Match level statistics 
could be like best teams or how the teams progressed during tournament. As we can see
in figure ~\ref{fig:concept} this was the high level design of the data we wanted to visualize 
for IPL T20 matches.

\begin{figure}[htbp]
\centering
\fbox{\includegraphics[width=\linewidth]{images/concept}}
\caption{Concept and Design Layout}
\label{fig:concept}
\end{figure}

\subsubsection{Visualization Methods selection}
For visualizing the data as described in figure ~\ref{fig:concept} we took some ideas
based on the charts shown during a live match. Here are some of the visualizations 
we chose to represent this data.
\begin{itemize}
\item location: location is always best represented by a map view, hence we chose 
to show the location of IPL matches on a map
\item tournament progress: this kind of data can be represented in a tabular form 
but doesn't become so intutive where you have to read into the details of the data.
It can also be represented in a tree format but representing date along with team would
make it clumsy. Hence, we chose a Gantt chart which is a perfect fit to represent 
how tournament progressed.
\item match progress: a live match shows the progress using a snail chart and we
got inspired by the snail chart to show this visualization. Snail chart shows the match 
progress using two line charts and also how wickets fell while runs being scored 
by two teams.
\item runs and wickets: these are two important aspects of a cricket match and we chose
to represent these using a boxplot which will provide us details on median, min, max along
with limits and team names for each season.
\item top bowler and batsman: we chose to use treemap which gives a good visualization
representing larger tile for top batsman and bowler. This will also give details of the team
to which this player belongs to.
\item dismissal: dismissal has different types like stumped, bowled, run out, caught,
hit wicket, lbw etc. A pie chart can represent the numbers better how a player bowls
and results in what kind of dismissals. Usually spin bowlers result in maximum caught
and fast bowlers result in lbw or bowled.
\item team performance: to choose the top team for a particular season or
multiple seasons we chose bar chart which gives good comparison or which teams
performed better in a tournament.
\item batsman performance: batsman performance is represented by strike rate(especially)
in a limited overs match like T20. For this form of cricket, we chose bar chart which gives 
a good comparison of strike rate of various players. Its is important to note that a player
can play for different team in different season, hence a bar chart becomes stacked chart for 
accommodating this scenario. 
\item runs scored: we chose to use bubble chart, since the size of bubble can clearly 
represent the high run scorers within a tournament or across multiple tournaments. 
\end{itemize}

\section{Technology Methods}
We used Tableau \cite{www-tableau} desktop to visualize this data. Tableau makes it easy to import different type
of datasources and visualizing the data easily using these imported data sources. There is a slight 
learning curve involved in understanding the tool and visualization methods. Some visualizations 
require creating calculated measures. We created these calculated measures like
\begin{itemize}
\item count of 6s
\item count of 4s
\item wickets
\item strike rate
\item dismissal
\end{itemize}

\section{Data sources}
We will be using datasources from IPL \cite{www-iplt20} official website
and Kaggle challenge \cite{www-kragle-iplc}.

\section{Results}
\subsubsection{Novelty and Insights}
Based on design we created following visualizations
\begin{itemize}
\item Map View: Map view shows location of IPL matches on a world map. Map view has
a associated dashboard which allows us to view team runs for each venue and wins by team 
per venue.
\item snail chart: snail chart shows match progress based on a match ID. It shows total runs on 
y-axis and overs on x-axis with two line charts for each team. On the line chart it shows the
wickets i.e. how players got dismissed. Line chart shows how the teams achieved their target.
\item winners chart: its a bar chart which show which team was ranked best for a given season.
You can select one or more years for this chart. The hover on the bar shows the rank of the team.
\item Toss Winner: this chart shows a comparison of the number of times team won the match vs 
number of times team won the toss. You can add a filter for a year.
\item Top Batsman: This chart is a treemap which shows a bigger tile for the team who has the 
top run scorer and then shows smaller tile for the other players in its team. This chart can be 
filtered by year. It gives a good idea of top run scorers for the season and the team.
\item Top Bowler: this chart is similar to top batsman where each tile shows the bowler
who took maximum number of wickets as well as the team to which he plays. We can filter
the data based on year of the tournament.
\item Bowling Extras: The chart represents the teams who gave lot of extras to their opponent.
This chart is a bubble chart and size of bubble represent the team which gave more number of 
extra runs. We can select one or more years to visualize this data.
\item Top Hitters: This chart represents the players who has big hitters in terms of 4s and 6s.
This is important chart since it tells which batsman were the most big hitters for a given season 
or multiple seasons. This chart is represented by a treemap.
\item Runs Scored: This chart represents the number of runs scored in a season. It shows median,
low, high watermarks. Also, it shows where each team lies  on the boxplot in terms of runs scored.
\item Wickets taken: Similar to runs scored chart, it shows number of wickets taken by each team
in the tournament. Also, it shows the median, min, max watermarks for the boxplot.
\item Match Progress: this chart is a Gantt's chart which shows the how teams played for a given 
season along with actual match dates. It shows when the teams played and how they reached to the
end of the tournament.
\item Wins By Venue: Chart shows the number of matches won by a team at a particular venue.
The filter to this chart is location. This is a bar chart.
\item Runs By Venue: Chart shows runs scored by each team on a particular venue. Its again a
bubble chart and size of bubble is proportional to the number of runs scored.
\item dismissal type: Its a pie chart and provides a distribution of type wickets taken by a bowler.
The wicket(dismissal) can be lbw, stumped, bowled, caught, run out, caught and bowled. It shows 
dismissal type for each bowler which indicates the bowlers strength in these areas.
\item Strike Rate: Indicates how fast a batsman scores runs in limited overs game and how powerful
as batsman is. This chart is bar chart and becomes stacked bar chart if a player played for multiple
IPL teams.
\item Man Of Match: This chart is a scatter plot with each dot on the plot indicates number of times
a player received man of match award and which team he belongs to. We can apply a year filter 
to this chart.
\item Extra Runs by Bowler: Its again a bubble chart and indicates number of extra runs for each bowl
bowled by the bowler. Each dot indicates the bowler and number of extra runs he conceeded and
number of ball he bowled. 
\end{itemize}
\subsubsection{craftsmanship and details}

\begin{figure}[htbp]
\centering
\fbox{\includegraphics[width=\linewidth]{images/dashboard}}
\caption{Charts By Venue}
\label{fig:dashboard}
\end{figure}

\begin{figure}[htbp]
\centering
\fbox{\includegraphics[width=\linewidth]{images/mapview}}
\caption{Map View}
\label{fig:mapview}
\end{figure}

\begin{figure}[htbp]
\centering
\fbox{\includegraphics[width=\linewidth]{images/snailChart}}
\caption{Snail Chart}
\label{fig:snailChart}
\end{figure}

\begin{figure}[htbp]
\centering
\fbox{\includegraphics[width=\linewidth]{images/winners}}
\caption{Winners}
\label{fig:winners}
\end{figure}

\begin{figure}[htbp]
\centering
\fbox{\includegraphics[width=\linewidth]{images/tossWinners}}
\caption{Toss Winners}
\label{fig:tossWinners}
\end{figure}

\begin{figure}[htbp]
\centering
\fbox{\includegraphics[width=\linewidth]{images/topBatsman}}
\caption{Top Batsman}
\label{fig:topBatsman}
\end{figure}

\begin{figure}[htbp]
\centering
\fbox{\includegraphics[width=\linewidth]{images/topBowler}}
\caption{Top Bowler}
\label{fig:topBowler}
\end{figure}

\begin{figure}[htbp]
\centering
\fbox{\includegraphics[width=\linewidth]{images/bowlingExtra}}
\caption{Bowling Extra}
\label{fig:bowlingExtra}
\end{figure}

\begin{figure}[htbp]
\centering
\fbox{\includegraphics[width=\linewidth]{images/topHitter}}
\caption{Top Hitter}
\label{fig:topHitter}
\end{figure}

\begin{figure}[htbp]
\centering
\fbox{\includegraphics[width=\linewidth]{images/runsScored}}
\caption{Runs Scored}
\label{fig:runsScored}
\end{figure}

\begin{figure}[htbp]
\centering
\fbox{\includegraphics[width=\linewidth]{images/wicketsTaken}}
\caption{Wickets Taken}
\label{fig:wicketsTaken}
\end{figure}

\begin{figure}[htbp]
\centering
\fbox{\includegraphics[width=\linewidth]{images/matchProgress}}
\caption{Match Progress}
\label{fig:matchProgress}
\end{figure}

\begin{figure}[htbp]
\centering
\fbox{\includegraphics[width=\linewidth]{images/winsByVenue}}
\caption{Wins By Venue}
\label{fig:winsByVenue}
\end{figure}

\begin{figure}[htbp]
\centering
\fbox{\includegraphics[width=\linewidth]{images/runsByVenue}}
\caption{Runs By Venue}
\label{fig:runsByVenue}
\end{figure}

\begin{figure}[htbp]
\centering
\fbox{\includegraphics[width=\linewidth]{images/dismissalType}}
\caption{Dismissal Types}
\label{fig:dismissalType}
\end{figure}

\begin{figure}[htbp]
\centering
\fbox{\includegraphics[width=\linewidth]{images/strikeRate}}
\caption{Strike Rate}
\label{fig:strikeRate}
\end{figure}

\begin{figure}[htbp]
\centering
\fbox{\includegraphics[width=\linewidth]{images/manOfMatch}}
\caption{Man Of the Match}
\label{fig:manOfMatch}
\end{figure}

\begin{figure}[htbp]
\centering
\fbox{\includegraphics[width=\linewidth]{images/extraRunsByBowler}}
\caption{Extra Runs By Bowler}
\label{fig:extraRunsByBowler}
\end{figure}
\FloatBarrier
\section{Discussion and Conclusion}
We faced few challenges while doing this project like
\begin{itemize}
\item tooling: we spent sometime in experimenting the right tool/technology.
We started off with a javascript based web application and hosting
it on a webserver. It was difficult to find a public hosted webserver as 
well as lot of learning involved in building a full fledge web application.
We also tried to put the data in a jupyter notebook using pyplot and pandas
but again hosting jupyter notebook was a challenge and getting user input in jupyter
notebook is not so intutive. Further we experimented with tableau and really
liked the tool which provides a public hosted app as well as facilitates different kind
of visualization libraries.
\item technology: Tableau \cite{www-tableau} was the tool we decided to go with but it involved some learning.
We looked at online tutorials and KB articles to get familiar with the tool. Tableau is pretty
rich in providing good collection of visualization libraries. 
\item hosting: With a custom application there was a challenge to 
host the application. However, tableau has a public web server
where different users can host their application. We hosted our application on 
\href{https://us-west-2b.online.tableau.com/#/site/summer2018dviz/views }{iplstats application}. ~\ref{fig:tableau_webserver}
\end{itemize} 
These visualizations for IPL data provide a different look to the data as well
as help the user to look at various trends in data. 
Another benefit these charts bring is aggregated view of data. With this view
one can see charts not just for one season but more than one season by selecting multiple
years. These charts provide good analysis of each season for various dimensions which is not covered 
by any of the sources explored above. It also visualizes the data using different visualization techniques
which make it more informative. This can be further extended  for providing trend analysis as well as doing 
some predictive analysis of IPL matches.

\begin{figure}[htbp]
\centering
\fbox{\includegraphics[width=\linewidth]{images/tableau_webserver}}
\caption{Project Hosted On Tableau Public Webserver}
\label{fig:tableau_webserver}
\end{figure}

\section{Acknowledgements}
 The authors thank Prof. YY Ahn for his technical guidance. The
 authors would also like to thank TAs of Data Visualization class for their valued
 support. 

\section{Repo} 
 All project and report document can be found at \href{https://github.com/abhishek8gupta/dviz-project-summer2018}{github project}.
 Project is hosted on \href{https://us-west-2b.online.tableau.com/#/site/summer2018dviz/views }{iplstats application}

\nocite{*}
\bibliographystyle{unsrt}
\bibliography{references}

\end{document}
